%%%%%%%%%%%%%%%%%%%%%%%%%%%%%%%%%%%%%%%%%%%
%%%              tZq                    %%%
%%%%%%%%%%%%%%%%%%%%%%%%%%%%%%%%%%%%%%%%%%%
\section[\tZq]{\tZq}
\label{subsec:tZq}

This section describes the MC samples used for the modelling of \tZq production.
\Cref{subsubsec:tZq_aMCP8} describes the \MGNLOPY[8] samples,

\subsection[MadGraph5\_aMC@NLO+Pythia8]{\MGNLOPY[8]}
\label{subsubsec:tZq_aMCP8}

\paragraph{Samples}
%\label{par:tZq_aMCP8_samples}

The descriptions below correspond to the samples in \cref{tab:tZq_aMCP8,tab:tZq_aMCP8_addRad}.

\begin{table}[htbp]
  \caption{Nominal \tZq samples produced with \MGNLOPY[8].}%
  \label{tab:tZq_aMCP8}
  \centering
  \begin{tabular}{l l}
    \toprule
    DSID range & Description \\
    \midrule
    412063 & \tZq \\
    \bottomrule
  \end{tabular}
\end{table}

\begin{table}[htbp]
  \caption{\tZq samples produced with \MGNLOPY[8] used to estimate initial-state radiation systematic uncertainties.}%
  \label{tab:tZq_aMCP8_addRad}
  \centering
  \begin{tabular}{l l}
    \toprule
    DSID range & Description \\
    \midrule
    412065 & \tZq, A14Var3c up \\
    410064 & \tZq, A14Var3c down \\
    \bottomrule
  \end{tabular}
\end{table}

\paragraph{Short description:}

The production of \tZq events was modelled using the \MGNLO[2.3.3]~\cite{Alwall:2014hca}
generator at NLO with the \NNPDF[3.0nlo]~\cite{Ball:2014uwa} parton distribution function~(PDF).
The events were interfaced with \PYTHIA[8.230]~\cite{Sjostrand:2014zea} using the A14 tune~\cite{ATL-PHYS-PUB-2014-021} and the
\NNPDF[2.3lo]~\cite{Ball:2014uwa} PDF set.

The uncertainty due to initial-state radiation (ISR) was estimated by comparing the nominal \tZq sample with two additional samples,
which had the same settings as the nominal one, but employed the Var3c up and down variations of the A14 tune.


\paragraph{Long description:}

The \tZq sample was simulated using the \MGNLO[2.3.3]~\cite{Alwall:2014hca}
generator at NLO with the \NNPDF[3.0nlo]~\cite{Ball:2014uwa} parton distribution function~(PDF). The events were interfaced with
\PYTHIA[8.230]~\cite{Sjostrand:2014zea} using the A14 tune~\cite{ATL-PHYS-PUB-2014-021}
and the \NNPDF[2.3lo]~\cite{Ball:2014uwa} PDF set. Off-resonance events away from the \(Z\) mass peak were included.
The top quark was decayed at LO using \MADSPIN~\cite{Frixione:2007zp,Artoisenet:2012st} to preserve spin correlations.
The four-flavour scheme was used, where all the quark masses are set to zero, except for the top and bottom quarks.
Following the discussion in Ref.~\cite{Frederix:2012dh}, the functional form of the renormalisation and factorisation scales
was set to \(4\sqrt{m_b^2+\pTX[2][b]}\), where the \(b\)-quark was the one produced by a gluon splitting in the event.
The decays of bottom and charm hadrons were simulated using the \EVTGEN program~\cite{Lange:2001uf}.

The \tZq total cross-section, calculated at next-to-leading order (NLO) using \MGNLO[2.3.3] with the \NNPDF[3.0nlo] PDF set,
is \(\qty{800}{\fb}\), with an uncertainty of \(^{+6.1}_{-7.4}\)\%. The uncertainty was computed by varying the renormalisation and
factorisation scales by a factor of two and by a factor of 0.5.

The uncertainty due to initial-state radiation (ISR) was estimated by comparing the nominal \tZq sample with two additional samples,
which have the same settings as the nominal one, but employed the Var3c up or down variation of the A14 tune, which
corresponds to the variation of \alphas for ISR in the A14 tune.

To evaluate the effect of renormalisation and factorisation scale uncertainties, the two scales were varied simultaneously by factors 2.0 and 0.5.
To evaluate the PDF uncertainties for the nominal PDF, the 100 variations for \NNPDF[2.3lo] were taken into account.
