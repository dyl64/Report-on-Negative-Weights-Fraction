% !TEX root = MC_snippets.tex

\chapter{Single-boson processes}
%\label{sec:VJ}

In the following paragraphs, the set-up of the current ATLAS single-boson baseline samples is described.
Details of the full process configuration are given in the PUB note~\cite{ATL-PHYS-PUB-2017-006}. In the case of \SHERPA samples,
a minimal description of built-in systematic uncertainties is also given.

%In addition, \SHERPA[2.1.1] is still used for cross checks and in the evaluation of theoretical uncertainties.
%Samples using this version have been generated with the \CT[10nlo] set of PDFs~\cite{Lai:2010vv}, in conjunction
%with the dedicated set of tuned parton-shower parameters developed by the \SHERPA authors for this generator version.

\section[Sherpa MEPS@NLO]{\SHERPA (\MEPSatNLO)}
%\label{sec:vjets-sherpa}

\subsection{QCD \(V+\)jets}
%\label{sec:vjets-sherpa-vjets}

\subsection*{Samples}
%\label{sec:vjets-sherpa-samples}

The descriptions below correspond to the samples in \cref{tab:vjets-sherpa}.

\begin{table}[!htbp]
  \caption{\(V\)+jets samples with \SHERPA.}%
  \label{tab:vjets-sherpa}
  \centering
  \begin{tabular}{l l}
    \toprule
    DSID range & Description \\
    \midrule
    364100--364113    & \(Z\to\mu\mu\)   \\
    364198--364203    & \(Z\to \mu\mu\) (\(\qty{10}{\GeV} < m_{\ell\ell} < \qty{40}{\GeV}\)) \\
    364359, 364362, 364281  &  \(Z\to\mu\mu\) (very low mass)   \\
    364114--364127    & \(Z\to ee\)    \\
    364204--364209    & \(Z\to ee\) (\(\qty{10}{\GeV} < m_{\ell\ell} < \qty{40}{\GeV}\)) \\
    364358, 364361, 364282 & \(Z\to ee\) (very low mass) \\
    364128--364141    & \(Z\to \tau\tau\)    \\
    364210--364215    & \(Z\to \tau\tau\) (\(\qty{10}{\GeV} < m_{\ell\ell} < \qty{40}{\GeV}\)) \\
    364282, 364360, 364363 &  \(Z\to \tau\tau\) \\
    364142--364155    & \(Z\to \nu\nu\)   \\
    364156--364169    & \(W\to \mu\nu\)   \\
    364170--364183    & \(W\to e\nu\)    \\
    364184--364197    & \(W\to \tau\nu\)  \\
    364216--364229    & \(Z\to\ell\ell,W\to\ell\nu\) (high \pT)\\
    \bottomrule
  \end{tabular}
\end{table}

\paragraph{Short description:}

The production of \(V+\)jets was simulated with the
\SHERPA[2.2.1]~\cite{Bothmann:2019yzt}
generator using next-to-leading-order (NLO) matrix elements (ME) for up to two partons, and leading-order (LO) matrix elements
for up to four partons calculated with the Comix~\cite{Gleisberg:2008fv}
and \OPENLOOPS~\cite{Buccioni:2019sur,Cascioli:2011va,Denner:2016kdg} libraries. They
were matched with the \SHERPA parton shower~\cite{Schumann:2007mg} using the \MEPSatNLO
prescription~\cite{Hoeche:2011fd,Hoeche:2012yf,Catani:2001cc,Hoeche:2009rj}
using the set of tuned parameters developed by the \SHERPA authors.
The \NNPDF[3.0nnlo] set of PDFs~\cite{Ball:2014uwa} was used and the samples
were normalised to a next-to-next-to-leading-order (NNLO)
prediction~\cite{Anastasiou:2003ds}.


\paragraph{Long description:}

The production of \(V+\)jets was simulated with the \SHERPA[2.2.1]~\cite{Bothmann:2019yzt}
generator. In this set-up, NLO-accurate matrix elements for up to two partons, and LO-accurate matrix elements for up
to four partons were calculated with the Comix~\cite{Gleisberg:2008fv} and
\OPENLOOPS~\cite{Buccioni:2019sur,Cascioli:2011va,Denner:2016kdg} libraries.
The default \SHERPA parton shower~\cite{Schumann:2007mg} based on
Catani--Seymour dipole factorisation and the cluster hadronisation model~\cite{Winter:2003tt}
were used. They employed the dedicated set of tuned parameters developed by the
\SHERPA authors and the \NNPDF[3.0nnlo] PDF set~\cite{Ball:2014uwa}.

The NLO matrix elements for a given jet multiplicity were matched to the parton
shower (PS) using a colour-exact variant of the MC@NLO
algorithm~\cite{Hoeche:2011fd}. Different jet multiplicities were then merged
into an inclusive sample using an improved CKKW matching
procedure~\cite{Catani:2001cc,Hoeche:2009rj} which was extended to NLO
accuracy using the \MEPSatNLO prescription~\cite{Hoeche:2012yf}. The merging threshold
was set to \qty{20}{\GeV}.

Uncertainties from missing higher orders were
evaluated~\cite{Bothmann:2016nao} using seven variations of the QCD
factorisation and renormalisation scales in the matrix elements by
factors of \(0.5\) and \(2\), avoiding variations in opposite directions.

Uncertainties in the nominal PDF set were evaluated using 100 replica
variations. Additionally, the results were cross-checked using the
central values of the \CT[14nnlo]~\cite{Dulat:2015mca} and
\MMHT[nnlo]~\cite{Harland-Lang:2014zoa} PDF sets. The effect of the uncertainty
in the strong coupling constant \(\alphas\) was assessed by variations of \(\pm 0.001\).






The \(V\)+jets samples were normalised to a next-to-next-to-leading-order (NNLO)
prediction~\cite{Anastasiou:2003ds}.


\subsection{Electroweak \(Vjj\) (VBF)}
%\label{sec:vjets-sherpa-vjj}

The descriptions below correspond to the samples in
\cref{tab:ewkvjets-sherpa}.  Samples include the VBF and \(V\)-strahlung diagrams, but
they do not include semileptonic \(VV\) diagrams and do not overlap with QCD \(V+\)jets samples.

\begin{table}[!htbp]
  \caption{Electroweak \(Vjj\) samples with \SHERPA.}%
  \label{tab:ewkvjets-sherpa}
  \centering
  \begin{tabular}{l l}
    \toprule
    DSID range & Description \\
    \midrule
    700358--700364 & EWK \(Vjj\) (baseline) \\
    308092--308096 & EWK \(Vjj\) (legacy) \\
    \bottomrule
  \end{tabular}
\end{table}

\paragraph{Description (baseline setups):}

Electroweak production of \(\ell\ell jj\), \(\ell\nu jj\) and \(\nu\nu jj\) final states
was simulated with \SHERPA[2.2.11]~\cite{Bothmann:2019yzt} using
leading-order (LO) matrix elements with up to one additional parton emission.
The matrix elements were merged with the \SHERPA parton
shower~\cite{Schumann:2007mg} following the \MEPSatLO
prescription~\cite{Catani:2001cc} and using the set of tuned
parameters developed by the \SHERPA authors.
The \NNPDF[3.0nnlo] set of
PDFs~\cite{Ball:2014uwa} was employed. The samples were produced
using the VBF approximation, which avoids overlap  with semileptonic
diboson topologies by requiring a \(t\)-channel colour-singlet exchange.
The starting conditions of the CS shower are set according to the 
large-\(N_c\) amplitudes supplied by Comix~\cite{Buckley:2021gfw} to achieve 
the correct VBF-appropriate radiation pattern.

%These samples are generated in the \(G_\mu\) scheme using, ensuring an
%optimal description of pure electroweak interactions at the
%electroweak scale.

\paragraph{Description (legacy setups):}

Electroweak production of \(\ell\ell jj\), \(\ell\nu jj\) and \(\nu\nu jj\) final states
was simulated with \SHERPA[2.2.1]~\cite{Bothmann:2019yzt} using
leading-order (LO) matrix elements with up to two additional parton emissions.
The matrix elements were merged with the \SHERPA parton
shower~\cite{Schumann:2007mg} following the \MEPSatLO
prescription~\cite{Catani:2001cc} and using the set of tuned
parameters developed by the \SHERPA authors.  The \NNPDF[3.0nnlo] set of
PDFs~\cite{Ball:2014uwa} was employed. The samples were produced
using the VBF approximation, which avoids overlap  with semileptonic
diboson topologies by requiring a \(t\)-channel colour-singlet exchange.

%These samples are generated in the \(G_\mu\) scheme using, ensuring an
%optimal description of pure electroweak interactions at the
%electroweak scale.


\section[MadGraph5 (CKKW-L)]{\MADGRAPH (CKKW-L)}
%\label{sec:vjets-mg5py8_ckkwl}

\subsection*{Samples}
%\label{sec:vjets-mg5py8_ckkwl-samples}

The descriptions below correspond to the samples in
\cref{tab:vjets-mg5py8_ckkwl}. The set-ups of \(N_\text{parton}\)- and
\HT-sliced samples differ slightly between the two slicing schemes
with regard to the matrix element PDF, the jet-clustering radius parameter
and the scale used in the evaluation of  \alphas to determine the weight of
each splitting. The short description merges the two set-ups and requires
the paper editors to select the appropriate PDF set (or gracefully describe
both); the long description is left unmerged.

\begin{table}[!htbp]
  \caption{\(V\)+jets samples with \MGPY[8] using CKKW-L merging.}%
  \label{tab:vjets-mg5py8_ckkwl}
  \centering
  \begin{tabular}{l l}
    \toprule
    DSID range & Description \\
    \midrule
    363123--363146 & \HT-sliced \(Z\to\mu\mu\)   \\
    363147--363170 & \HT-sliced \(Z\to ee\)     \\
    361510--361514 & \(N_\text{parton}\)-sliced \(Z\to\tau\tau\) \\
    361515--361519 & \(N_\text{parton}\)-sliced \(Z\to\nu\nu\)   \\
    363624--363647 & \HT-sliced \(W\to \mu\nu\)   \\
    363600--363623 & \HT-sliced \(W\to e\nu\)    \\
    363648--363671 & \HT-sliced \(W\to\tau\nu\)  \\
    \bottomrule
  \end{tabular}
\end{table}

\paragraph{Short description for \HT-sliced and \(N_\text{parton}\)-sliced \(V\)+jets:}

QCD \(V\)+jets production was simulated with \MGNLO[2.2.2]~\cite{Alwall:2014hca},
using LO-accurate matrix elements (ME) with up to four final-state partons.
The ME calculation employed the \NNPDF[3.0nlo] set of PDFs~\cite{Ball:2014uwa}
(\HT-sliced) / \NNPDF[2.3lo] set of PDFs~\cite{Ball:2012cx} (\(N_\text{parton}\)-sliced).
Events were interfaced to \PYTHIA[8.186]~\cite{Sjostrand:2007gs} for the modelling
of the parton shower, hadronisation, and  underlying event. The overlap between
matrix element and parton shower emissions was removed using the CKKW-L
merging procedure~\cite{Lonnblad:2001iq,Lonnblad:2011xx}. The A14
tune~\cite{ATL-PHYS-PUB-2014-021} of \PYTHIA[8] was used with the
\NNPDF[2.3lo] PDF set~\cite{Ball:2012cx}.
The decays of bottom and charm
hadrons were performed by \EVTGEN[1.2.0]~\cite{Lange:2001uf}.
The \(V\)+jets samples were normalised to a next-to-next-to-leading-order (NNLO)
prediction~\cite{Anastasiou:2003ds}.


\paragraph{\HT-sliced long description:}

QCD \(V\)+jets production was simulated with LO-accurate matrix elements (ME)
for up to four partons with \MGNLO[2.2.2]~\cite{Alwall:2014hca}. The ME calculation was interfaced with
\PYTHIA[8.186]~\cite{Sjostrand:2007gs} for the modelling of the parton
shower, hadronisation, and underlying event. To remove overlap between the matrix
element and the parton shower the CKKW-L merging
procedure~\cite{Lonnblad:2001iq,Lonnblad:2011xx} was applied with a
merging scale of \qty{30}{\GeV} and a jet-clustering radius parameter of
\(0.2\). In order to better model the region of large jet \pT, the
strong coupling constant \alphas was evaluated at the scale of each splitting to
determine the weight. The matrix element calculation was performed with
the \NNPDF[3.0nlo] PDF set~\cite{Ball:2014uwa} with \(\alphas= 0.118\). The calculation was done
in the five-flavour number scheme with massless \(b\)- and
\(c\)-quarks. Quark masses were reinstated in the \PYTHIA[8] parton shower.
The renormalisation and factorisation scales were set to the \MADGRAPH default
values, based on a clustering of the event. The A14
tune~\cite{ATL-PHYS-PUB-2014-021} of \PYTHIA[8] was used with the
\NNPDF[2.3lo] PDF set~\cite{Ball:2012cx} with \(\alphas=0.13\).
The decays of bottom and charm hadrons were performed by \EVTGEN[1.2.0]~\cite{Lange:2001uf}.


\paragraph{\(N_\text{parton}\)-sliced long description:}

QCD \(V\)+jets production was simulated with LO-accurate matrix elements (ME) for up to four partons with
\MGNLO[2.2.2]~\cite{Alwall:2014hca}. The ME calculation was interfaced
with \PYTHIA[8.186]~\cite{Sjostrand:2007gs} for the modelling of the parton
shower and underlying event. To remove overlap between the matrix
element and the parton shower the CKKW-L merging
procedure~\cite{Lonnblad:2001iq,Lonnblad:2011xx} was applied with a
merging scale of \qty{30}{\GeV} and a jet-clustering radius parameter of
\(0.4\). In order to better model the region of large jet \pT, the
strong coupling constant \alphas was evaluated at the scale of each splitting to
determine the weight. The matrix element calculation was performed with
the \NNPDF[2.3lo] PDF set~\cite{Ball:2012cx} with \(\alphas= 0.13\). The calculation was done
in the five-flavour number scheme with massless \(b\)- and
\(c\)-quarks. Quark masses were reinstated in the \PYTHIA[8] parton shower.
The renormalisation and factorisation scales were set to the \MADGRAPH default
values, based on a clustering of the event. The A14 tune~\cite{ATL-PHYS-PUB-2014-021}
of \PYTHIA[8] was used with the \NNPDF[2.3lo] PDF set~\cite{Ball:2012cx} with \(\alphas=0.13\).
The decays of bottom and charm hadrons were performed by \EVTGEN[1.2.0]~\cite{Lange:2001uf}.

% No usable aMC@NLO \FXFX samples exist in ATLAS central production, therefore
% the description is commented out to avoid confusion

%% \section{\MGNLO (\FXFX)}
%% \label{sec:vjets-mg5-fxfx}

%% \begin{table}[!htbp]
%% \caption{Samples with \MGNLOPYTHIA[8] using CKKW-L merging.}%
%% \label{tab:mg5-fxfx}
%% \centering
%% \begin{tabular}{l l}
%% \toprule
%% DSID range & Description \\
%% \midrule
%% 999999-888888 & V+jets HT-sliced\\
%% \bottomrule
%% \end{tabular}
%% \end{center}
%% \end{table}

%% Samples have also been generated using the \MGNLO program to
%% generate matrix elements for \(V\) + 0, 1 and 2 partons at NLO accuracy.
%% The showering and subsequent hadronisation has been performed using \PYTHIA[8.210] with the A14 tune, using the \NNPDF[2.3lo]
%% PDF set with \(\alphas = 0.13\).
%% The different jet multiplicities are merged using the
%% \FXFX prescription~\cite{Frederix:2012ps} implemented in the \MGNLO program
%% (version 2.3.3 is used here).

%% \MGNLO performs a 5FNS calculation with massless \(b\)- and \(c\)-quarks in the matrix element, and massive quarks in the
%% \PYTHIA shower.
%% The PDF input set in use for event generation is \NNPDF[2.3nlo] PDF set with \(\alphas = 0.119\)
%% and the samples have been generated with additional weights for the PDF replicas as well as scale variations of
%% the renormalisation and the factorisation scales~\cite{Frederix:2011ss}.

%% The impact of various merging scales (\muQ) has been studied,
%% analysing three different values: \qty{20}{\GeV} (down variation), \qty{25}{\GeV} (nominal value) and \qty{50}{\GeV} (up variation).
%% At the event-generation level, the minimum jet transverse momentum, \(\pT^j\) is required to be at least
%% \qty{8}{\GeV} with no restriction on the absolute value of the jet pseudorapidity (jet \(|\eta|\)).
%% %The cut on the \(\pT^j\) should be at least half the value used for the merging scale,
%% %in order to not introduce any bias in the MC generation.
%% The samples have been generated using LHAPDF-6.1.5~\cite{Butterworth:2014efa} and FastJet-3.1.0~\cite{Cacciari:2006sm}.


\section[Inclusive Powheg \(V\)]{Inclusive \POWHEG \(V\)}
%\label{sec:v-powheg}

\subsection{QCD \(V+\)jets}
%\label{sec:v-powheg-vjets}

The descriptions below correspond to the samples in \cref{tab:v-powheg}.

\begin{table}[!htbp]
  \caption{Inclusive \(V\) samples with \POWHEG.}%
  \label{tab:v-powheg}
  \centering
  \begin{tabular}{l l}
    \toprule
    DSID range & Description \\
    \midrule
    361100--361108    & \(W^+,W^-,Z/\gamma^\ast\) with \(e,\mu,\tau\) decays\\
    301000--301178, 344722    & high-mass slices: \(W^+,W^-,Z\) with \(e,\mu,\tau\) decays \\
    361664--361669 & \(Z/\gamma^\ast\) low-mass slices (\(m=6\)--10--60\,\unit{\GeV})\\
    426335--426336 & \(Z/\gamma^\ast\) high-\(\pTX[][\ell\ell] > \qty{150}{\GeV}\) slices \\
    \bottomrule
  \end{tabular}
\end{table}

\paragraph{Description:}

The \POWHEGBOX[v1] MC generator~\cite{Nason:2004rx,Frixione:2007vw,Alioli:2010xd,Alioli:2008gx}
was used for the simulation at NLO accuracy of the hard-scattering processes of \(W\)
and \(Z\) boson production and decay in the electron, muon, and \(\tau\)-lepton
channels. It was interfaced to \PYTHIA[8.186]~\cite{Sjostrand:2007gs}
for the modelling of the parton shower, hadronisation, and underlying
event, with parameters set according to the \AZNLO
tune~\cite{STDM-2012-23}. The \CT[10nlo] PDF set~\cite{Lai:2010vv} was used
for the hard-scattering processes, whereas the \CTEQ[6L1] PDF
set~\cite{Pumplin:2002vw} was used for the parton shower. The effect of
QED final-state radiation was simulated with \PHOTOSpp[3.52]~\cite{Golonka:2005pn,Davidson:2010ew}.
The \EVTGEN[1.2.0] program~\cite{Lange:2001uf} was used to decay bottom and charm hadrons.

% No usable \POWHEG \MINLO samples exist in ATLAS central production, therefore
% the description is commented out to avoid confusion

%% \section{\POWHEG \MINLO}
%% \label{sec:vjets-minlo}

%% Predictions from \POWHEG \MINLO~\cite{Alioli:2010xd,Hamilton:2012np,Hamilton:2012rf}
%% interfaced to \PYTHIA[8.210]~\cite{Sjostrand:2007gs} with the
%% AZNLO tune~\cite{STDM-2012-23} are obtained to produce
%% \(V\)+jets events, and will be referred to in the remainder of the note
%% as \POWHEG \MINLO+\PYTHIA[8]. The \EVTGEN[1.2.0] program~\cite{Lange:2001uf}
%% is used as an afterburner to better simulate the decays of bottom and charm hadrons.
%% The \PHOTOSpp[3.61] program~\cite{Golonka:2005pn,Davidson:2010ew} is interfaced to \PYTHIA to accurately
%% describe the QED final state radiation, and the emission of lepton pairs is activated in this setup.
%% The PDF set used in \POWHEG is \CT[14nnlo]~\cite{Dulat:2015mca} whereas the PDF set used in the parton shower
%% is the \CTEQ[6L1]~\cite{Pumplin:2002vw} leading order set.

%% \POWHEG \MINLO is an improvement of the CKKW matching procedure to reach NLO accuracy for
%% boson production in association one jet, and smoothly
%% merges to NLO single boson production at low boson \pT, using no external merging scale.
%% NLO contributions from the CKKW Sudakov form factors are subtracted to the virtual corrections provided by \POWHEGBOX.
%% Thanks to the optimal scale and to the Sudakov form factors, the
%% \POWHEG \MINLO samples do not need any Born suppression cut, and their cross-sections are well-behaved at small boson \pT,
%% contrary to fixed order predictions that diverge in this kinematic phase space.

\subsection{Electroweak \(Vjj\) (VBF)}
%\label{sec:v-powheg-vjj}

The descriptions below correspond to the samples in
\cref{tab:ewkvjets-pp8}.  Samples include the VBF and \(V\)-strahlung diagrams, but
they do not include semileptonic \(VV\) diagrams and do not overlap with the QCD \(V+\)jets samples.

\begin{table}[!htbp]
  \caption{Electroweak \(Vjj\) samples with \POWHEG.}%
  \label{tab:ewkvjets-pp8}
  \centering
  \begin{tabular}{l l}
    \toprule
    DSID range & Description \\
    \midrule
    600931--600939 & EWK \(Vjj\) \\
    \bottomrule
  \end{tabular}
\end{table}

\paragraph{Description:}

Electroweak production of \(\ell\ell jj\) and \(\ell\nu jj\) final states
was simulated with 
\POWHEGBOX[v2]~\cite{Frixione:2007nw,Nason:2004rx,Frixione:2007vw,Alioli:2010xd}
using the \NNPDF[3.0nlo]~\cite{Ball:2014uwa} parton distribution functions (PDF)
and is accurate to next-to-leading order (NLO) in perturbative QCD. The sample was produced with
the VBF approximation, which requires a $t$-channel colour-singlet exchange to remove overlap with
diboson topologies~\cite{Jager:2012xk,Schissler:2013nga}. The parton-level events were passed to \PYTHIA[8.245] 
to add parton-showering hadronisation and underlying-event activity, using the A14~\cite{ATL-PHYS-PUB-2014-021} 
set of tuned parameters. 
The correct VBF-appropriate radiation pattern was achieved by using the dipole-recoil option.
The \EVTGEN[1.7.0] program~\cite{Lange:2001uf} was used for the properties of the bottom and charm hadron decays.

%These samples are generated in the \(G_\mu\) scheme using, ensuring an
%optimal description of pure electroweak interactions at the
%electroweak scale.

